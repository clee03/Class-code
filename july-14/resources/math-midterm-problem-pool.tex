\documentclass{article}
\usepackage[utf8]{inputenc}
\usepackage{amsmath}
\usepackage{amssymb}
\usepackage{hyperref}
\newcommand*{\Perm}[2]{{}^{#1}\!P_{#2}}%

\title{Discrete Math midterm problem pool}
\date{\today}
\author{CS161}
\begin{document}
\maketitle


\section{Simple counting}
\begin{enumerate}
\item \( 2^3 * 2^5 = \)
\item \( \dfrac{2^{16}}{2^{12}} = \)
\item \( \dbinom{13}{4} = \)
\item \( \dbinom{13}{13} = \)
\item \( \Perm{4}{4} = \)
\item \( \Perm{4}{3} = \)
\item \( \Perm{4}{2} = \)
\item \( \Perm{13}{4} = \)
\item \( \dbinom{13}{4} = \dfrac{\Perm{x}{y}}{\Perm{w}{z}} \): what are x, y, w, and z?
\end{enumerate}


\section{Counting}
\begin{enumerate}
\item A restaurant has 5 predefined meals, 3 drink options and 4 food options. You want to get either a predefined meal, or a food-drink pair. How many different ways might you order?

\item I have 8 shirts, one of which is gray. I have 3 pairs of pants, one of which is gray. How many shirt-pants combinations can I make that involve a gray article of clothing?

\item How many strings of 2 digits do not have a nine in them?

\item License plates have three uppercase letters followed by three digits. How many different license plates are there that include a nine? Hint: first calculate how many three digit strings include a nine.

\item How many different license plates are there that don't include a nine?

\item How many different license plates are there that don't have any letter appear more than once and don't have any digit appear more than once?

\item How many different license plates are there that don't have any letter appear more than once \textit{or} don't have any digit appear more than once?

\item If all students have one of five majors and at least one of three minors, how many students are needed in a room to guarantee that at least three of them in the room share a major-minor combination? (ie. three people majoring in astronomy with a minor in physical education)

\item How many ways are there to select 10 countries in the United Nations to serve on a council if three are selected from a block of 30, four are selected from a block of 55, and the others are selected from the remaining 65 countries?

\item A group of 14 friends put in a pizza order for pick-up. Somebody has to go pick up the pizza. But nobody wants to go alone and no pair of people want to go as just the two of them. At least three people need to go on the pizza run. How many different subsets of people could go on the pizza run?

\item A department has 7 men 10 women. How many ways are there to form a committee of five members if it must have more women than men?

\item How many permutations of the letters ABCDEFG have C and D right next to each other? (ie. CD or DC) Equivalently, if you are taking photos of a wedding party of three bridesmaids, two groomsmen, and the happy couple, how many ways can you arrange the people from left to right in the photo given that the bride and groom must stand next to each other?
\end{enumerate}


\section{Advanced Counting}
\begin{enumerate}
\item How many non-negative integers less than 1000 (when written with no leading zeros, but 0 is still written as 0):
  \begin{enumerate}
  \item have exactly three digits?
  \item have an odd number of digits?
  \item have at least one digit equal to 9?
  \item have no odd digits?
  \item have two consecutive fives?
  \item are palindromes? (ex. 1, 11, 101)
  \end{enumerate}

\item There are four books, called A, B, C, D. How many ways can you put these books on a shelf such that A is to the left of C?

\item How many distinct five card hands could you draw from a normal 52 card deck that have a four-of-a-kind? (ie. four twos and some other card, or four kings and some other card)

\item Find $n$ if
  \begin{enumerate}
  \item $\Perm{n}{2} = 110$
  \item $\Perm{n}{n} = 5040$
  \item $\Perm{n}{4} = 12 * \Perm{n}{2}$
  \end{enumerate}

\item Find $n$ if
  \begin{enumerate}
  \item $ \dbinom{n}{2} = 45$
  \item $ \dbinom{n}{3} = \Perm{n}{2}$
  \item $ \dbinom{n}{5}= \dbinom{n}{2}$
  \end{enumerate}

\item If the numbers from 1 to 1000 are written out on a piece of paper, how many 9's appear on that paper?
\end{enumerate}


\section{Series Summations}
\begin{enumerate}
\item Prove that \( 1 + 2 + ... + n \) can be calculated with \( \dfrac{n(n+1)}{2} \)

\item Prove that $ 1 + 4 + 7 + ... + (3n - 2) $ can be calculated with $ \dfrac{n(3n - 1)}{2} $

\item Show that \( 1^3+2^3+...+n^3= \Big( \dfrac{n(n+1)}{2} \Big)^2 \) for every positive integer n.

\item Show that \( 1*1! + 2*2!+...+n*n! = (n + 1)! - 1 \) for every positive integer n.

\item Show that \( 1^2 + 3^2 + 5^2 + ... + (2n + 1)^2=  \Big( \dfrac{(n + 1)(2n + 1)(2n + 3)}{3} \Big)^2 \) for every non-negative integer n.

\item Show that the sum of the first $n$ even positive integers is \( n(n + 1) \).

\item Prove that for all positive integers, $n$:
\[ \frac{2}{3} + \frac{2}{9} + \frac{2}{27} + ... + \frac{2}{3^n} = 1 - \frac{1}{3^n} \]
\end{enumerate}


\section{Problems involving inequality}
\begin{enumerate}
\item Prove that for any integer $n$ greater than 6, $n!$ is greater than $3^n$.
\[n! > 3^n \mid n > 6 \]

\item Prove that for all integers \( n \geq 4 \) that \( n^2 \leq n! \)

\item Prove that for all integers \( n > 1 \) that \( n! < n^n \)
\end{enumerate}


\section{Problems involving divisibility}
\begin{enumerate}
\item Prove that for any integer $n$ greater than or equal to 0, $3^{2n} - 1$ is divisible by 8.
\[ 3^{2n} - 1 \equiv 0 \mod 8 \]

\item Prove that 3 evenly divides \( n^3 + 2n \) for every non-negative integer n.

\item Prove that 6 evenly divides \( n^3 - n \) for every non-negative integer n.
\end{enumerate}


\section{Problems from other domains}
\begin{enumerate}
\item Prove that all postage amounts greater than 7 cents can be made with combinations of 3 and 5 cent stamps.

\item Prove that every amount of postage of 6 cents or higher can be formed using just 2 cent and 5 cent stamps.

\item Write a method in Java that calculates the nth fibonacci number recursively. (a common interview question) fib(0) is 0 and fib(1) is 1.

\item Write a method in Java that calculates factorial of a number n recursively.
\end{enumerate}


\end{document}
